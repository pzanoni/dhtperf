\documentclass[compress]{beamer}

%\usetheme{Warsaw}
\usetheme{Singapore}
%\usecolortheme{rose}
%\useinnertheme{rounded}

\usepackage[utf8]{inputenc}
\usepackage[brazil]{babel}

%\usepackage{listings}

\newcommand{\dhtperf}{\texttt{dhtperf}}

\title[\dhtperf{} tutorial]{\dhtperf{} tutorial\\Performance evaluation for
\emph{real} Distributed Hash Tables}
\author{Paulo Ricardo Zanoni}
\institute{UFPR -- Universidade Federal do Paraná}
\date{2010}


\AtBeginSection[]
{
    \begin{frame}<beamer>
        \frametitle{Contents}
        \tableofcontents[currentsection]
    \end{frame}
}

\begin{document}

\begin{frame}
\titlepage
\end{frame}

%\begin{frame}{Contents}
%\tableofcontents
%\end{frame}

\begin{frame}{Disclaimer}
This documentation is a little bit OLD. It still describes most of dhtperf, but
some details might have changed.
\end{frame}

\section{Introduction}
\begin{frame}{What is \dhtperf?}
\dhtperf{} is:
\begin{itemize}
    \item a framework for measuring DHT performance
    \item intended for real-world DHTs (i.e., \emph{not} simulation)
    \item a tool for applying \emph{workloads} to DHTs and obtaining
    \emph{metrics}
    \item a \emph{wrapper} for existing DHT software
    \item extensible software
    \item free software (GPLv3+)
\end{itemize}
\end{frame}

\begin{frame}{Design goals}
The software should be...
\begin{itemize}
    \item extensibile: easy to add new DHT backends, workload profiles and
    new metrics
    \item modular: extending is as easy as implementing a class
    \item available: free to download and modify as in GPLv3+
    \item scalable: as scalable as the backend DHT
    \item usable: documented, unlike some DHT simulators
    \item portable: whenever possible, the code uses portable functions
    (although I only test on Linux)
    \item non-intrusive: \emph{wrap} and \emph{use} existing DHT software,
    don't modify
\end{itemize}
\end{frame}

\begin{frame}{The little pieces}
\begin{itemize}
    \item \texttt{dhtperf-master:} controls the performance tests, coordinates
    the controllers, one instance per test
    \item \texttt{dhtperf-controller:} controls a DHT node, collects the metrics
    data, one instance per DHT node
    \item \emph{the protocol:} messages exchanged between the
    \texttt{dhtperf-master} and the \texttt{dhtperf-controllers}
    \item \emph{workload file:} contains the number of nodes, duration of the
    tests and one or more \emph{node profiles}
    \item \emph{node profile:} describes how a node should behave during the
    performance test
    \item \emph{metrics:} things to be measured by the \emph{controller} while
    the test is running
\end{itemize}
\end{frame}

\begin{frame}{Figure}
(insert cool image containing the dhtperf-master, dhtperf-controller, the
protocol, the workload file, and the node profiles)
\end{frame}

%\begin{frame}{}
%\begin{itemize}
%\end{itemize}
%\end{frame}
\section{Using}

\subsection{Obtaining and installing}
\begin{frame}{Obtaining}
You'll need:
\begin{itemize}
    \item the latest release from \url{http://foo.bar} or a \texttt{git
    clone}\footnote{\url{http://git-scm.com}}
    of the development snapshot from \url{git://gitorious.org/foobarbaz}
    \item a C++ compiler
    \item cmake \footnote{\url{http://www.cmake.org}}
    \item libboost \footnote{\url{http://www.boost.org}}
    \item latex-beamer if you want to build the documentation
    \footnote{\url{http://bitbucket.org/rivanvx/beamer/}}
\end{itemize}
\end{frame}

\begin{frame}{Installing}
To compile:
\begin{itemize}
    \item \texttt{cd} into the source
    \item \texttt{mkdir build; cd build}
    \item \texttt{cmake ..}
    \item \texttt{make}
\end{itemize}
For more building options, check the README file.
\end{frame}

\subsection{Running}
\begin{frame}{Running}
Short version:
\begin{itemize}
    \item write the \emph{workload file}
    \item \texttt{./dhtperf-master -w workload-file}
    \item for each DHT node, \texttt{./dhtperf-controller -d dhtUsed
    masterHostName}
    \item watch the output
    \item \texttt{ls dhtperf-results}
\end{itemize}
\end{frame}

\subsection{The workload file}
\begin{frame}{Workload file}
nodes 5\\
duration 60\\
\#\\
profile prof1\\
\# profile definition goes here, see the next slide\\
profile prof2\\
\# (...)\\
profile prof3\\
\# (...)\\
\# associate nodes with profiles:\\
default-profile prof1\\
node 0 prof2\\
node 2 prof3
\end{frame}

\begin{frame}{Profiles}
\begin{columns}[c]
\column{0.5\textwidth}
profile prof1\\
join exact 10\\
leave exact 30\\
put none\\
get exact 20 key 25 key2 \textbackslash \\
26 key3 28 key
\column{0.5\textwidth}
profile prof2\\
join exact 3\\
leave exact 35\\
put exact 6 key value\\
get poisson 3 4 34\\
\end{columns}
\end{frame}

\begin{frame}{Profiles II}
Profiles:
\begin{itemize}
    \item every \emph{node profile} has 4 profiles: \texttt{join},
    \texttt{leave}, \texttt{put} and \texttt{get}
    \item every profile has a name (\texttt{exact}, \texttt{none},
    \texttt{poisson}, etc.) and as many arguments as it needs
    \item check the documentation for the arguments
    \item the idea is that it should be easy to create new profiles: just
    implement a new class that inherits from another, filling the necessary
    functions
\end{itemize}
\end{frame}

\subsection{Running the commands}

\begin{frame}{Running the commands}
First you'll need to start the master:
\begin{itemize}
    \item \texttt{./dhtperf-master -w workload-file}
\end{itemize}

Then you should start every controller. We recommend a script that does
\texttt{ssh} and runs the command.
\begin{itemize}
    \item \texttt{./dhtperf-controller -d dhtUsed masterHostName}
\end{itemize}

The controllers will then contact the masters and do everything on their own.

It \emph{is} possible to run the controllers in the same machine as the master.
You can call that \emph{emulation}.
\end{frame}

\subsection{Checking the output}
\begin{frame}{Checking the output}
After the test ends the controllers will pass the data to the master, who
will process it and write the results into the \texttt{dhtperf-output}
directory. Check it and enjoy =)
\end{frame}

\section{The workflow}
\begin{frame}{The workflow}
\begin{itemize}
    \item the \texttt{dhtperf-master} is a single-threaded TCP server
    \item after you launch it, it parses the workload file and then starts
    accepting connections
    \item when you start a \texttt{dhtperf-controller}, it tries to open a TCP
    connection with the \emph{master}
    \item the first node to connect the master will be \emph{node $0$} and so
    on (for the purpose of assigning profiles)
    \item after connecting, the \emph{controller} sends a
    \texttt{GetProfileQuery} message, which is replied with a
    \texttt{GetProfileReply} message.
\end{itemize}
\end{frame}
\begin{frame}{The workflow II}
\begin{itemize}
    \item after every node has started and sent a \texttt{GetProfileQuery}, the
    \emph{master} sends a \texttt{Start} message to every node, starting the
    performance test
    \item no more messages are exchanged until the end of the execution
    % XXX: cabe mais aqui!
\end{itemize}
\end{frame}

\begin{frame}{The workflow III}
\begin{itemize}
    \item TODO
\end{itemize}
\end{frame}

\section{Hacking}

\subsection{The protocol}
\begin{frame}{The protocol}
The \texttt{protocol.h} file contains the protocol definitions:
\begin{itemize}
    \item every protocol message is a class that inherits the
    \texttt{dhtperfproto::message::Message} base class
    \item every message is made of two parts: the header and the body
    \item the header contains only the opcode and the size of the body
    \item the body's contents is specific to each message
    \item the protocol is binary, not plain-text
\end{itemize}
\end{frame}

\subsection{Adding a new DHT}
\begin{frame}{Adding a new DHT}
The code that deals with specific DHTs is inside \texttt{controller/dhtnodes/}.
\begin{itemize}
    \item every DHT node inherits the DhtNode class
    \item the DhtNode class already has some useful code, so the inherited
    class only needs to implement a few functions
    \item check the DummyDht class for an easy example
\end{itemize}
\end{frame}

\subsection{Adding a new profile}
\begin{frame}{Adding a new profile}
\begin{itemize}
    \item to add a new profile, you'll have to create a class that inherits from
    one of: \texttt{JoinProfile}, \texttt{LeaveProfile}, \texttt{PutProfile} or
    \texttt{GetProfile} (see the code inside \texttt{controller/profiles/})
    \item note that all of these classes inherit from the
    \texttt{GenericProfile} class
    \item internally, every class has an \texttt{EventQueue} of one of the
    \texttt{DhtEvent} classes (see the \texttt{events.h} file)
\end{itemize}
\end{frame}

\subsection{Adding a new metric}
\begin{frame}{Adding a new metric}
TODO
\end{frame}

\section{Conclusion}
\begin{frame}{Conclusion}
Now that you understand a little bit about \dhtperf{}, try to run it.

Start with the included workload files, they're small.

Wrote a patch? Has a contribution? A complaint? Is the code horribly unusable?
Please send feedback! Write to \url{paulo@c3sl.ufpr.br}.
\end{frame}

\end{document}
